\documentclass[12pt, a4paper, article, UTF8]{ctexart}
\usepackage[subpreambles=true]{standalone}
\usepackage{import}
\begin{document}
\section{前言}

本文撰写的目的是为了让我重拾机器人时不再需要从头开始看,这样太浪费时间了。因此要写一篇目的是为了让自己看懂的教程。具体来说,之后当我长时间没看robotics的时候,我希望我能在24小时内能通过这篇东西将我的记忆全部刷新,而不是每次都要重新再学一遍。

笔者在研究其他教材的过程中,发现了现有的教程或文献或多或少有以下问题:

1. 一上来就给个概念或者定义。概念一旦有了,便开始一些更不知道是从哪里冒出来的引理,证明了这些概念,然后继续推进进度。一切看起来好像非常像模像样很是一回事。读者即不知道这些概念是为了什么,从何而来,有什么用。

2. 很多物理量的定义和推导过程非常依赖代数,缺乏intuition和数形结合,或者缺乏实际的物理意义。

针对问题1:笔者认为,概念是怎么来的,为什么服务,为什么如此构建比如何证明要重要得多。证明虽然很重要,证明是验证正确性的唯一标准,但是提出概念是比证明更有意义。

针对问题2:robotics的核心是rigid body dynamics,rigid body dynamics中的大部分物理量都应该是有实际意义的,可以在现实世界中具体化的,而非一些抽象的代数。这些物理意义拆碎揉烂,分解出来,应该是高中生都能理解的现象。

具有问题1和问题2的教材在阅读过程当中,很容易在一个又一个概念当中迷失,不知道一些概念用来做什么,于是在学习的过程中,把所有变量抽象化,代数上证明通了就通了,不去关心基本的intuition和细节。因此在实际使用的时候(基本上是通过代码复现来进行数值计算),便无法获得很多宝贵的insight从而失去了发现问题或者解决问题的机会。而且,由于演绎的过程抽象化,不符合人脑的直觉思维模式,经过一段时间后,必然会忘记很多内容。这些问题,在笔者学习过程中是真实遇到的。

因此笔者写这份讲义是为了解决以上两个问题,以及解决两个问题造成的后果。笔者的思路是:站在一个什么都不知道但是学过高中物理、微积分、线性代数基础的人的视角,从零开始建立rigid body dynamics直至robot的dynamics model。在这过程中,我们不能假设自己构建的东西这些东西最终会引导正确的结论,因为在摸索的过程当中,我们应该是不知道最终的终点的。因此要上述仅有的知识的基础上一步步解释为何要如此构建,怎么来的。并且我们重视物理量、证明过程与intuition、实际物理意义的结合。我们可以使用已有的数学工具和分析方法帮助验证我们的思路,但我们应当假设自己对rigid body dynamics的认知水平停留在17世纪,对如何建立模型不带有任何事后诸葛亮的眼光。当然这说着容易,实际上不免会收到已经有结论的影响,所以笔者也只能尽力而为之。


\end{document}