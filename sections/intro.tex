\documentclass[12pt, a4paper, article, UTF8]{ctexart}
\usepackage[subpreambles=true]{standalone}
\usepackage{import}
\begin{document}
\section{讲义概述}
我们为什么要研究robotics?我想这个问题应该不需要我来回答。你打开这本讲义的时候,应该自己心里有一定的答案。Robot的形态和种类是很丰富的,其中每一种都值得单独写一本书或很多本书。我们研究的重心是多关节机器人,比如industrial robot、limbed robot,其中包括open-chain robot和包含close close的robot。当然,无论形态如何变化,robot一定是能运动的,是能产生motion的,是能对外做功的。

既然涉及到运动,做功,那么我们就要研究其运动和做功的规律。这就涉及到了dynamics。Dynamics是研究一切运动及其变化的基础。我们研究的robot dynamics不是柔性机构,而是具有high rigidity的rigid body dynamics。Rigid body在运动和做功的过程当中,不会发生形变。

Robot的dynamics到最后总会转化为一种canonical form,即
\begin{align}
\tau = M(\theta)\ddot{\theta} + C(\theta,\dot{\theta})+g(\theta) \label{eq:canonicaldynamics}
\end{align}
这说白了,就是更加复杂的牛顿第二定律,即$F=ma$,其中$\tau$为input force,$M(\theta)$为mass,$\ddot{\theta}$为acceleration,$C(\theta,\dot{\theta})$为由于non-inertial frame造成的Coriolis force,$g(\theta)$为gravity。

我们研究rigid body dynamics,就是为了能够准确地描述$F=ma$,了解force与input force、position、velocity、acceleration的之间的相互关系以便我们可以将这个模型运用到后续的control和planning。
\end{document}